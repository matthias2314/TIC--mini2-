\newpage
\section{Conclusión}
\noindent Las actividades desarrolladas permitieron reforzar de manera práctica los conceptos vistos en el curso, especialmente el uso de la Raspberry Pi junto con sensores, botones e interfaces gráficas. En la primera actividad se logró crear un conjunto de minijuegos variados y un sistema de comunicación sencillo pero funcional, lo que ayudó a comprender mejor cómo organizar programas más grandes y cómo registrar información de forma estructurada. \newline

\noindent En la segunda actividad, la Pokédex permitió profundizar en el diseño de interfaces y en la forma de mostrar información de manera ordenada, además de incluir elementos físicos como un LED que responde al tipo de Pokémon. Esto hizo posible ver cómo la programación puede interactuar directamente con el hardware, dando lugar a aplicaciones más completas e interesantes. \newline

\noindent En conjunto, ambos proyectos ayudaron a desarrollar habilidades de programación, creatividad y resolución de problemas, demostrando que la Raspberry Pi es una herramienta muy útil para aprender y experimentar con ideas nuevas.
