\newpage
%#####################################################
\anexossection{Información Adicional}
%#####################################################
\anexossubsection{Tipos de LEDs}
%#####################################################
\anexossubsubsection{RGB LEDs}
¡Los LED RGB (Rojo-Verde-Azul) son en realidad tres LED en uno! Pero eso no significa que solo pueda hacer tres colores. Debido a que el rojo, el verde y el azul son los colores primarios aditivos, puede controlar la intensidad de cada uno para crear todos los colores del arco iris. La mayoría de los LED RGB tienen cuatro pines: uno para cada color y un pin común. En algunos, el pin común es el ánodo, y en otros, es el cátodo.

\begin{figure}[ht]
    \centering
    \includegraphics[width=0.2\textwidth]{Img/led_RGB.png}
    \caption{LED de cátodo transparente común RGB.  Fuente: }
    \label{fig:fondos_random1}
\end{figure}

%#####################################################
\anexossubsubsection{LEDs con circuitos integrados}
Algunos LED son más inteligentes que otros. Tomemos el LED de ciclismo, por ejemplo. Dentro de estos LED, en realidad hay un circuito integrado que  permite que el LED parpadee sin ningún controlador externo. Aquí hay un primer plano del IC (el gran chip cuadrado negro en la punta del yunque) que controla los colores.

¡Simplemente enciéndalo y míralo ir! Estos son excelentes para proyectos en los que desea un poco más de acción, pero no tiene espacio para circuitos de control. ¡Incluso hay LED parpadeantes RGB que recorren miles de colores!


\begin{figure}[ht]
    \centering
    \includegraphics[width=0.2\textwidth]{Img/led_IC.png}
    \caption{Primer plano del LED de ciclo lento de 5 mm.  Fuente: } 
    \label{fig:fondos_random2}
\end{figure}

%#####################################################
\anexossubsubsection{RGB LEDs}
Los LED SMD no son tanto un tipo específico de LED sino un tipo de paquete. A medida que la electrónica se hace cada vez más pequeña, los fabricantes han descubierto cómo meter más componentes en un espacio más pequeño. Las piezas SMD (dispositivo de montaje en superficie) son versiones pequeñas de sus contrapartes estándar. Aquí hay un primer plano de un LED direccionable WS2812B empaquetado en un pequeño paquete 5050.
 
Los LED SMD vienen en varios tamaños, ¡desde bastante grandes hasta más pequeños que un grano de arroz! Debido a que son tan pequeños y tienen almohadillas en lugar de piernas, no son tan fáciles de trabajar, pero si tiene poco espacio, podrían ser justo lo que recetó el médico.


\begin{figure}[ht]
    \centering
    \includegraphics[width=0.2\textwidth]{Img/led_SMD.png}
    \caption{ Primer plano direccionable WS2812B.  Fuente: }
    \label{fig:fondos_random3}
\end{figure}
