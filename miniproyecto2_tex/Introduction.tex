\newpage
\section{Introducción}
\noindent En este informe se presentan dos actividades prácticas realizadas utilizando una Raspberry Pi, con el objetivo de aprender a combinar programación, electrónica básica y el uso de sensores y botones. A lo largo del proyecto se desarrollaron distintos minijuegos, interfaces gráficas y sistemas simples de comunicación entre dispositivos. \newline

\noindent En la primera actividad, TIC is Among Us, se implementaron varios minijuegos hechos tanto en consola como con sensores físicos, como un joystick, botones y otros módulos del kit. Además, se creó un sistema de comunicación entre un jugador y un host central usando archivos de registro, permitiendo coordinar rondas, resultados y acciones dentro del juego. Esto permitió trabajar de manera práctica temas como lectura de sensores, creación de interfaces, uso de temporizadores y organización de eventos. \newline

\noindent La segunda actividad consistió en desarrollar una Pokédex interactiva, diseñada con PyQt6 y programada para mostrar imágenes, información de distintos Pokémon y controlar un LED que cambia de color según su tipo. Este proyecto permitió trabajar la carga de datos desde archivos, el diseño de interfaces y la conexión entre la parte visual y el hardware. \newline

\noindent Ambas actividades permitieron aplicar lo aprendido de forma concreta, mezclando software y hardware para crear sistemas interactivos y funcionales.
